is task 2's part satisfying the requirements here?
Faculty of Media Engineering and Technology    
German University in Cairo 
Dr. Nourhan Ehab 
Presentation and Evaluation Guideline 
Milestone 3 marks the final and most technically demanding stage of the project, where 
each team must demonstrate a fully integrated and functional Graph-RAG system. To 
ensure a fair and consistent assessment, the evaluation for Milestone 3 will follow a 
structured format and will be conducted per team over a 45-minute slot. 
The evaluation consists of two parts: 
1. Team Presentation (Deliverables Evaluation) 15% 
The team will present the implemented system, focusing strictly on the system 
architecture, pipeline integration, retrieval strategies, experimental results, and 
the live demo. This part evaluates the functionality, completeness, and 
performance of your Graph-RAG pipeline. The presentation should take a 
minimum of 18 minutes and a maximum of 22 minutes. 
2. Individual Q&A Evaluation 15% 
Each team member will be examined individually on their specific component. 
Depending on the remaining time after the presentation, the Q&A may include 
multiple rounds per person. You may be asked to walk through portions of your 
code, explain the sequence of operations within your component, discuss error 
cases based on your experiments, or justify implementation decisions. 
When a question is directed to a specific member, only that member is 
allowed to answer. 
This document outlines the required presentation structure, expectations for both parts 
of the evaluation, and the rules that will be applied. Please read it carefully and prepare 
your work accordingly to ensure a smooth and efficient evaluation process. 
Faculty of Media Engineering and Technology    
German University in Cairo 
Dr. Nourhan Ehab 
Presentation Outline: 
● High-Level System Architecture (2 minutes) 
○ Show an overview of your pipeline 
○ Present your task of choice, and if you used an external dataset. 
● Input Preprocessing  (2 minutes) 
○ Intent classifier (rule-based, LLM-based, or hybrid) 
○ Entity extraction, with examples 
○ Embedding step (if used) 
● Graph Retrieval Layer - Baseline (2-3 minutes) 
○ Show your Cypher query templates (at least 10 should be 
implemented) 
○ Show snippet of retrieved nodes/relationships 
● Graph Retrieval Layer - Embedding-Based Retrieval (2-3 minutes) 
○ State the approach you selected: Node embeddings OR Feature vector 
embeddings 
○ Show the two embedding models compared, with experiment results 
● LLM Layer (3-4 minutes) 
○ Context Construction, how you integrate the input, baseline Cypher 
query output, and the embedding output. 
○ Prompt Structure 
○ LLMs Comparison; experiments should include quantitative metrics and 
qualitative evaluation. 
● Error Analysis & Improvements (2 minutes) 
● Live Demo (4-5 minutes) 
○ Start by wrapping up the full pipeline, from raw input how to get to the final 
answer. 
○ The demo should be live not recorded video 
○ We should be able to switch between the embedding model, and switch 
between the LLMs  
○ Using your chosen questions to answers, we evaluate the integration of 
your pipeline 
○ The demo must show integration. not isolated components. 
○ The UI should be reflecting the process done in the background, check the 
Build UI section in the project description. 
Faculty of Media Engineering and Technology    
German University in Cairo 
Dr. Nourhan Ehab 
What not to do: 
● Do NOT explain concepts that were explained in the lab 
● Do NOT add introductions, problem statements, motivations, or related 
work/framework descriptions. 
● In your slides, Do NOT depend only on diagrams, only on text, only on 
screenshots 
● No text-heavy descriptions.Highlight only what you actually implemented. 
● Do NOT include dataset descriptions or high-level overviews of the theme. 
Regulations: 
● Each team member is responsible for one component of the milestone. You will 
be required to indicate who is responsible for which component when scheduling 
your evaluation. (Make sure to distribute the work equally between team 
members (e.g. doing the slides of the presentation is NOT an individual task)) 
● All team members must participate in the presentation. 
● Presentation slides must be submitted along with the milestone deliverables by 
December 15th at 23:59. 
● You must work and present as a unified team with a fully integrated, functional 
project. (Randomly assigned teams included.) 
● If one or more members are unable to attend or present, this should be 
communicated from the start of the evaluation time so that the remaining 
members may present their parts to avoid losing the deliverable grades. 
Late Policy: 
We respect your time, especially during revision week, and we expect you to respect 
other teams’ time as well. Delays during MS1 caused significant disruptions to the 
schedule; this will not be tolerated in the MS3 evaluation. 
● Being 19 minutes late or more will result in cancellation of your evaluation 
and a grade of 0 for both the deliverables and the evaluation. 
● Being 15-18 minutes late will result in a 30% deduction from your final grade. 
● Being 10-14 minutes late will result in a 15% deduction from your final grade. 
Faculty of Media Engineering and Technology    
German University in Cairo 
Dr. Nourhan Ehab 
Please keep in mind that your evaluation slot will end at the scheduled time. Arriving 
late reduces the number of rounds available for each team member and decreases your 
opportunity to correct any mistakes during the individual evaluation. 
